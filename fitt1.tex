\begin{deluxetable}{lllllll}
\tablecolumns{5}
\tablewidth{0pc}
\tablecaption{Parameter Grid )  
\label{fitt1}
}
\tablehead{

\colhead{} & \colhead{\teff}& \colhead{Mass}& \colhead{\menv} & \colhead{\mhe} & \colhead{\xo} & \qfm  \\
\colhead{}&\colhead{(K)} &\colhead{\msun}&\colhead{log(\mstar)} &\colhead{log(\mstar)}&
\colhead{log(\mstar)} & \colhead{log(\mstar)} \\
}
\startdata
Minimum  & 21{,}000 & 0.500 & $2.00$ & $4.00$ & 0.10  & 0.10  \\%             &                       \\
Maximum  & 26{,}000 & 0.700 & $3.40$ & $7.00$ & 1.00  & 0.80  \\%             &                       \\
Step size & 200     & 0.010 & 0.20   & 0.20   & 0.10  & 0.05  \\%             &                       \\
%\multicolumn{8}{c}{Best fit parameters}\\
%                       & 21{,}200      & 0.675                         & $-3.20$       & $-5.60$       & 0.65          & 0.52                  & 3.12 s        \\


\enddata
\tablecaption{Region of parameter space covered by the master grid and best fit parameters (inclusive). 
\menv ~is the location of the base of the mixed helium and carbon layer, \mhe that of the base 
of the pure helium layer,\xo the oxygen abundance in the center and \qfm the location of the 
edge of the homogeneous C/O core.}
\end{deluxetable}
